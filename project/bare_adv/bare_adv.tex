
%% bare_adv.tex
%% V1.4
%% 2012/12/27
%% by Michael Shell
%% See: 
%% http://www.michaelshell.org/
%% for current contact information.
%%
%% This is a skeleton file demonstrating the advanced use of IEEEtran.cls
%% (requires IEEEtran.cls version 1.8 or later) with an IEEE Computer
%% Society journal paper.
%%
%% Support sites:
%% http://www.michaelshell.org/tex/ieeetran/
%% http://www.ctan.org/tex-archive/macros/latex/contrib/IEEEtran/
%% and
%% http://www.ieee.org/

%%*************************************************************************
%% Legal Notice:
%% This code is offered as-is without any warranty either expressed or
%% implied; without even the implied warranty of MERCHANTABILITY or
%% FITNESS FOR A PARTICULAR PURPOSE! 
%% User assumes all risk.
%% In no event shall IEEE or any contributor to this code be liable for
%% any damages or losses, including, but not limited to, incidental, %% consequential, or any other damages, resulting from the use or misuse
%% of any information contained here.
%%
%% All comments are the opinions of their respective authors and are not
%% necessarily endorsed by the IEEE.
%%
%% This work is distributed under the LaTeX Project Public License (LPPL)
%% ( http://www.latex-project.org/ ) version 1.3, and may be freely used, %% distributed and modified. A copy of the LPPL, version 1.3, is included
%% in the base LaTeX documentation of all distributions of LaTeX released
%% 2003/12/01 or later.
%% Retain all contribution notices and credits.
%% ** Modified files should be clearly indicated as such, including **
%% ** renaming them and changing author support contact information. **
%%
%% File list of work: IEEEtran.cls, IEEEtran_HOWTO.pdf, bare_adv.tex, %%          bare_conf.tex, bare_jrnl.tex, bare_jrnl_compsoc.tex, %%          bare_jrnl_transmag.tex
%%*************************************************************************

% *** Authors should verify (and, if needed, correct) their LaTeX system ***
% *** with the testflow diagnostic prior to trusting their LaTeX platform ***
% *** with production work. IEEE's font choices can trigger bugs that do ***
% *** not appear when using other class files.              ***
% The testflow support page is at:
% http://www.michaelshell.org/tex/testflow/



% IEEEtran V1.7 and later provides for these CLASSINPUT macros to allow the
% user to reprogram some IEEEtran.cls defaults if needed. These settings
% override the internal defaults of IEEEtran.cls regardless of which class
% options are used. Do not use these unless you have good reason to do so as
% they can result in nonIEEE compliant documents. User beware. ;)
%
%\newcommand{\CLASSINPUTbaselinestretch}{1.0} % baselinestretch
%\newcommand{\CLASSINPUTinnersidemargin}{1in} % inner side margin
%\newcommand{\CLASSINPUToutersidemargin}{1in} % outer side margin
%\newcommand{\CLASSINPUTtoptextmargin}{1in}  % top text margin
%\newcommand{\CLASSINPUTbottomtextmargin}{1in}% bottom text margin



% Note that the a4paper option is mainly intended so that authors in
% countries using A4 can easily print to A4 and see how their papers will
% look in print - the typesetting of the document will not typically be
% affected with changes in paper size (but the bottom and side margins will).
% Use the testflow package mentioned above to verify correct handling of
% both paper sizes by the user's LaTeX system.
%
% Also note that the "draftcls" or "draftclsnofoot", not "draft", option
% should be used if it is desired that the figures are to be displayed in
% draft mode.
%
\documentclass[12pt,journal,compsoc]{IEEEtran}
% The Computer Society requires 12pt.
% If IEEEtran.cls has not been installed into the LaTeX system files, % manually specify the path to it like:
% \documentclass[10pt,journal,compsoc]{../sty/IEEEtran}


% For Computer Society journals, IEEEtran defaults to the use of 
% Palatino/Palladio as is done in IEEE Computer Society journals.
% To go back to Times Roman, you can use this code:
%\renewcommand{\rmdefault}{ptm}\selectfont





% Some very useful LaTeX packages include:
% (uncomment the ones you want to load)



% *** MISC UTILITY PACKAGES ***
%
%\usepackage{ifpdf}
% Heiko Oberdiek's ifpdf.sty is very useful if you need conditional
% compilation based on whether the output is pdf or dvi.
% usage:
% \ifpdf
%  % pdf code
% \else
%  % dvi code
% \fi
% The latest version of ifpdf.sty can be obtained from:
% http://www.ctan.org/tex-archive/macros/latex/contrib/oberdiek/
% Also, note that IEEEtran.cls V1.7 and later provides a builtin
% \ifCLASSINFOpdf conditional that works the same way.
% When switching from latex to pdflatex and vice-versa, the compiler may
% have to be run twice to clear warning/error messages.






% *** CITATION PACKAGES ***
%
\ifCLASSOPTIONcompsoc
 % IEEE Computer Society needs nocompress option
 % requires cite.sty v4.0 or later (November 2003)
 % \usepackage[nocompress]{cite}
\else
 % normal IEEE
 % \usepackage{cite}
\fi
% cite.sty was written by Donald Arseneau
% V1.6 and later of IEEEtran pre-defines the format of the cite.sty package
% \cite{} output to follow that of IEEE. Loading the cite package will
% result in citation numbers being automatically sorted and properly
% "compressed/ranged". e.g., [1], [9], [2], [7], [5], [6] without using
% cite.sty will become [1], [2], [5]--[7], [9] using cite.sty. cite.sty's
% \cite will automatically add leading space, if needed. Use cite.sty's
% noadjust option (cite.sty V3.8 and later) if you want to turn this off
% such as if a citation ever needs to be enclosed in parenthesis.
% cite.sty is already installed on most LaTeX systems. Be sure and use
% version 4.0 (2003-05-27) and later if using hyperref.sty. cite.sty does
% not currently provide for hyperlinked citations.
% The latest version can be obtained at:
% http://www.ctan.org/tex-archive/macros/latex/contrib/cite/
% The documentation is contained in the cite.sty file itself.
%
% Note that some packages require special options to format as the Computer
% Society requires. In particular, Computer Society papers do not use
% compressed citation ranges as is done in typical IEEE papers
% (e.g., [1]-[4]). Instead, they list every citation separately in order
% (e.g., [1], [2], [3], [4]). To get the latter we need to load the cite
% package with the nocompress option which is supported by cite.sty v4.0
% and later.
%
% Note also the use of a CLASSOPTION conditional provided by 
% IEEEtran.cls V1.7 and later.





% *** GRAPHICS RELATED PACKAGES ***
%
\ifCLASSINFOpdf
 % \usepackage[pdftex]{graphicx}
 % declare the path(s) where your graphic files are
 % \graphicspath{{../pdf/}{../jpeg/}}
 % and their extensions so you won't have to specify these with
 % every instance of \includegraphics
 % \DeclareGraphicsExtensions{.pdf,.jpeg,.png}
\else
 % or other class option (dvipsone, dvipdf, if not using dvips). graphicx
 % will default to the driver specified in the system graphics.cfg if no
 % driver is specified.
 % \usepackage[dvips]{graphicx}
 % declare the path(s) where your graphic files are
 % \graphicspath{{../eps/}}
 % and their extensions so you won't have to specify these with
 % every instance of \includegraphics
 % \DeclareGraphicsExtensions{.eps}
\fi
% graphicx was written by David Carlisle and Sebastian Rahtz. It is
% required if you want graphics, photos, etc. graphicx.sty is already
% installed on most LaTeX systems. The latest version and documentation
% can be obtained at: 
% http://www.ctan.org/tex-archive/macros/latex/required/graphics/
% Another good source of documentation is "Using Imported Graphics in
% LaTeX2e" by Keith Reckdahl which can be found at:
% http://www.ctan.org/tex-archive/info/epslatex/
%
% latex, and pdflatex in dvi mode, support graphics in encapsulated
% postscript (.eps) format. pdflatex in pdf mode supports graphics
% in .pdf, .jpeg, .png and .mps (metapost) formats. Users should ensure
% that all non-photo figures use a vector format (.eps, .pdf, .mps) and
% not a bitmapped formats (.jpeg, .png). IEEE frowns on bitmapped formats
% which can result in "jaggedy"/blurry rendering of lines and letters as
% well as large increases in file sizes.
%
% You can find documentation about the pdfTeX application at:
% http://www.tug.org/applications/pdftex





% *** MATH PACKAGES ***
%
%\usepackage[cmex10]{amsmath}
% A popular package from the American Mathematical Society that provides
% many useful and powerful commands for dealing with mathematics. If using
% it, be sure to load this package with the cmex10 option to ensure that
% only type 1 fonts will utilized at all point sizes. Without this option, % it is possible that some math symbols, particularly those within
% footnotes, will be rendered in bitmap form which will result in a
% document that can not be IEEE Xplore compliant!
%
% Also, note that the amsmath package sets \interdisplaylinepenalty to 10000
% thus preventing page breaks from occurring within multiline equations. Use:
%\interdisplaylinepenalty=2500
% after loading amsmath to restore such page breaks as IEEEtran.cls normally
% does. amsmath.sty is already installed on most LaTeX systems. The latest
% version and documentation can be obtained at:
% http://www.ctan.org/tex-archive/macros/latex/required/amslatex/math/





% *** SPECIALIZED LIST PACKAGES ***
%\usepackage{acronym}
% acronym.sty was written by Tobias Oetiker. This package provides tools for
% managing documents with large numbers of acronyms. (You don't *have* to
% use this package - unless you have a lot of acronyms, you may feel that
% such package management of them is bit of an overkill.)
% Do note that the acronym environment (which lists acronyms) will have a
% problem when used under IEEEtran.cls because acronym.sty relies on the
% description list environment - which IEEEtran.cls has customized for
% producing IEEE style lists. A workaround is to declared the longest
% label width via the IEEEtran.cls \IEEEiedlistdecl global control:
%
% \renewcommand{\IEEEiedlistdecl}{\IEEEsetlabelwidth{SONET}}
% \begin{acronym}
%
% \end{acronym}
% \renewcommand{\IEEEiedlistdecl}{\relax}% remember to reset \IEEEiedlistdecl
%
% instead of using the acronym environment's optional argument.
% The latest version and documentation can be obtained at:
% http://www.ctan.org/tex-archive/macros/latex/contrib/acronym/


%\usepackage{algorithmic}
% algorithmic.sty was written by Peter Williams and Rogerio Brito.
% This package provides an algorithmic environment fo describing algorithms.
% You can use the algorithmic environment in-text or within a figure
% environment to provide for a floating algorithm. Do NOT use the algorithm
% floating environment provided by algorithm.sty (by the same authors) or
% algorithm2e.sty (by Christophe Fiorio) as IEEE does not use dedicated
% algorithm float types and packages that provide these will not provide
% correct IEEE style captions. The latest version and documentation of
% algorithmic.sty can be obtained at:
% http://www.ctan.org/tex-archive/macros/latex/contrib/algorithms/
% There is also a support site at:
% http://algorithms.berlios.de/index.html
% Also of interest may be the (relatively newer and more customizable)
% algorithmicx.sty package by Szasz Janos:
% http://www.ctan.org/tex-archive/macros/latex/contrib/algorithmicx/




% *** ALIGNMENT PACKAGES ***
%
%\usepackage{array}
% Frank Mittelbach's and David Carlisle's array.sty patches and improves
% the standard LaTeX2e array and tabular environments to provide better
% appearance and additional user controls. As the default LaTeX2e table
% generation code is lacking to the point of almost being broken with
% respect to the quality of the end results, all users are strongly
% advised to use an enhanced (at the very least that provided by array.sty)
% set of table tools. array.sty is already installed on most systems. The
% latest version and documentation can be obtained at:
% http://www.ctan.org/tex-archive/macros/latex/required/tools/


%\usepackage{mdwmath}
%\usepackage{mdwtab}
% Also highly recommended is Mark Wooding's extremely powerful MDW tools, % especially mdwmath.sty and mdwtab.sty which are used to format equations
% and tables, respectively. The MDWtools set is already installed on most
% LaTeX systems. The lastest version and documentation is available at:
% http://www.ctan.org/tex-archive/macros/latex/contrib/mdwtools/


% IEEEtran contains the IEEEeqnarray family of commands that can be used to
% generate multiline equations as well as matrices, tables, etc., of high
% quality.


%\usepackage{eqparbox}
% Also of notable interest is Scott Pakin's eqparbox package for creating
% (automatically sized) equal width boxes - aka "natural width parboxes".
% Available at:
% http://www.ctan.org/tex-archive/macros/latex/contrib/eqparbox/




% *** SUBFIGURE PACKAGES ***
%\ifCLASSOPTIONcompsoc
% \usepackage[caption=false,font=normalsize,labelfont=sf,textfont=sf]{subfig}
%\else
% \usepackage[caption=false,font=footnotesize]{subfig}
%\fi
% subfig.sty, written by Steven Douglas Cochran, is the modern replacement
% for subfigure.sty, the latter of which is no longer maintained and is
% incompatible with some LaTeX packages including fixltx2e. However, % subfig.sty requires and automatically loads Axel Sommerfeldt's caption.sty
% which will override IEEEtran.cls' handling of captions and this will result
% in non-IEEE style figure/table captions. To prevent this problem, be sure
% and invoke subfig.sty's "caption=false" package option (available since
% subfig.sty version 1.3, 2005/06/28) as this is will preserve IEEEtran.cls
% handling of captions.
% Note that the Computer Society format requires a larger sans serif font
% than the serif footnote size font used in traditional IEEE formatting
% and thus the need to invoke different subfig.sty package options depending
% on whether compsoc mode has been enabled.
%
% The latest version and documentation of subfig.sty can be obtained at:
% http://www.ctan.org/tex-archive/macros/latex/contrib/subfig/




% *** FLOAT PACKAGES ***
%
%\usepackage{fixltx2e}
% fixltx2e, the successor to the earlier fix2col.sty, was written by
% Frank Mittelbach and David Carlisle. This package corrects a few problems
% in the LaTeX2e kernel, the most notable of which is that in current
% LaTeX2e releases, the ordering of single and double column floats is not
% guaranteed to be preserved. Thus, an unpatched LaTeX2e can allow a
% single column figure to be placed prior to an earlier double column
% figure. The latest version and documentation can be found at:
% http://www.ctan.org/tex-archive/macros/latex/base/


%\usepackage{stfloats}
% stfloats.sty was written by Sigitas Tolusis. This package gives LaTeX2e
% the ability to do double column floats at the bottom of the page as well
% as the top. (e.g., "\begin{figure*}[!b]" is not normally possible in
% LaTeX2e). It also provides a command:
%\fnbelowfloat
% to enable the placement of footnotes below bottom floats (the standard
% LaTeX2e kernel puts them above bottom floats). This is an invasive package
% which rewrites many portions of the LaTeX2e float routines. It may not work
% with other packages that modify the LaTeX2e float routines. The latest
% version and documentation can be obtained at:
% http://www.ctan.org/tex-archive/macros/latex/contrib/sttools/
% Do not use the stfloats baselinefloat ability as IEEE does not allow
% \baselineskip to stretch. Authors submitting work to the IEEE should note
% that IEEE rarely uses double column equations and that authors should try
% to avoid such use. Do not be tempted to use the cuted.sty or midfloat.sty
% packages (also by Sigitas Tolusis) as IEEE does not format its papers in
% such ways.
% Do not attempt to use stfloats with fixltx2e as they are incompatible.
% Instead, use Morten Hogholm'a dblfloatfix which combines the features
% of both fixltx2e and stfloats:
%
% \usepackage{dblfloatfix}
% The latest version can be found at:
% http://www.ctan.org/tex-archive/macros/latex/contrib/dblfloatfix/


%\ifCLASSOPTIONcaptionsoff
% \usepackage[nomarkers]{endfloat}
% \let\MYoriglatexcaption\caption
% \renewcommand{\caption}[2][\relax]{\MYoriglatexcaption[#2]{#2}}
%\fi
% endfloat.sty was written by James Darrell McCauley, Jeff Goldberg and 
% Axel Sommerfeldt. This package may be useful when used in conjunction with 
% IEEEtran.cls' captionsoff option. Some IEEE journals/societies require that
% submissions have lists of figures/tables at the end of the paper and that
% figures/tables without any captions are placed on a page by themselves at
% the end of the document. If needed, the draftcls IEEEtran class option or
% \CLASSINPUTbaselinestretch interface can be used to increase the line
% spacing as well. Be sure and use the nomarkers option of endfloat to
% prevent endfloat from "marking" where the figures would have been placed
% in the text. The two hack lines of code above are a slight modification of
% that suggested by in the endfloat docs (section 8.4.1) to ensure that
% the full captions always appear in the list of figures/tables - even if
% the user used the short optional argument of \caption[]{}.
% IEEE papers do not typically make use of \caption[]'s optional argument, % so this should not be an issue. A similar trick can be used to disable
% captions of packages such as subfig.sty that lack options to turn off
% the subcaptions:
% For subfig.sty:
% \let\MYorigsubfloat\subfloat
% \renewcommand{\subfloat}[2][\relax]{\MYorigsubfloat[]{#2}}
% However, the above trick will not work if both optional arguments of
% the \subfloat command are used. Furthermore, there needs to be a
% description of each subfigure *somewhere* and endfloat does not add
% subfigure captions to its list of figures. Thus, the best approach is to
% avoid the use of subfigure captions (many IEEE journals avoid them anyway)
% and instead reference/explain all the subfigures within the main caption.
% The latest version of endfloat.sty and its documentation can obtained at:
% http://www.ctan.org/tex-archive/macros/latex/contrib/endfloat/
%
% The IEEEtran \ifCLASSOPTIONcaptionsoff conditional can also be used
% later in the document, say, to conditionally put the References on a 
% page by themselves.





% *** PDF, URL AND HYPERLINK PACKAGES ***
%
%\usepackage{url}
% url.sty was written by Donald Arseneau. It provides better support for
% handling and breaking URLs. url.sty is already installed on most LaTeX
% systems. The latest version and documentation can be obtained at:
% http://www.ctan.org/tex-archive/macros/latex/contrib/url/
% Basically, \url{my_url_here}.


% NOTE: PDF thumbnail features are not required in IEEE papers
%    and their use requires extra complexity and work.
%\ifCLASSINFOpdf
% \usepackage[pdftex]{thumbpdf}
%\else
% \usepackage[dvips]{thumbpdf}
%\fi
% thumbpdf.sty and its companion Perl utility were written by Heiko Oberdiek.
% It allows the user a way to produce PDF documents that contain fancy
% thumbnail images of each of the pages (which tools like acrobat reader can
% utilize). This is possible even when using dvi->ps->pdf workflow if the
% correct thumbpdf driver options are used. thumbpdf.sty incorporates the
% file containing the PDF thumbnail information (filename.tpm is used with
% dvips, filename.tpt is used with pdftex, where filename is the base name of
% your tex document) into the final ps or pdf output document. An external
% utility, the thumbpdf *Perl script* is needed to make these .tpm or .tpt
% thumbnail files from a .ps or .pdf version of the document (which obviously
% does not yet contain pdf thumbnails). Thus, one does a:
% 
% thumbpdf filename.pdf 
%
% to make a filename.tpt, and:
%
% thumbpdf --mode dvips filename.ps
%
% to make a filename.tpm which will then be loaded into the document by
% thumbpdf.sty the NEXT time the document is compiled (by pdflatex or
% latex->dvips->ps2pdf). Users must be careful to regenerate the .tpt and/or
% .tpm files if the main document changes and then to recompile the
% document to incorporate the revised thumbnails to ensure that thumbnails
% match the actual pages. It is easy to forget to do this!
% 
% Unix systems come with a Perl interpreter. However, MS Windows users
% will usually have to install a Perl interpreter so that the thumbpdf
% script can be run. The Ghostscript PS/PDF interpreter is also required.
% See the thumbpdf docs for details. The latest version and documentation
% can be obtained at.
% http://www.ctan.org/tex-archive/support/thumbpdf/


% NOTE: PDF hyperlink and bookmark features are not required in IEEE
%    papers and their use requires extra complexity and work.
% *** IF USING HYPERREF BE SURE AND CHANGE THE EXAMPLE PDF ***
% *** TITLE/SUBJECT/AUTHOR/KEYWORDS INFO BELOW!!      ***
\newcommand\MYhyperrefoptions{bookmarks=true,bookmarksnumbered=true, pdfpagemode={UseOutlines},plainpages=false,pdfpagelabels=true, colorlinks=true,linkcolor={black},citecolor={black},urlcolor={black}, pdftitle={Memora: A Lifelog Search Engine},%<!CHANGE!
pdfsubject={Typesetting},%<!CHANGE!
pdfauthor={Blake Elias},%<!CHANGE!
pdfkeywords={lifelog, scene recognition, classification, search engine}}%<^!CHANGE!
%\ifCLASSINFOpdf
%\usepackage[\MYhyperrefoptions,pdftex]{hyperref}
%\else
%\usepackage[\MYhyperrefoptions,breaklinks=true,dvips]{hyperref}
%\usepackage{breakurl}
%\fi
% One significant drawback of using hyperref under DVI output is that the
% LaTeX compiler cannot break URLs across lines or pages as can be done
% under pdfLaTeX's PDF output via the hyperref pdftex driver. This is
% probably the single most important capability distinction between the
% DVI and PDF output. Perhaps surprisingly, all the other PDF features
% (PDF bookmarks, thumbnails, etc.) can be preserved in
% .tex->.dvi->.ps->.pdf workflow if the respective packages/scripts are
% loaded/invoked with the correct driver options (dvips, etc.). 
% As most IEEE papers use URLs sparingly (mainly in the references), this
% may not be as big an issue as with other publications.
%
% That said, Vilar Camara Neto created his breakurl.sty package which
% permits hyperref to easily break URLs even in dvi mode.
% Note that breakurl, unlike most other packages, must be loaded
% AFTER hyperref. The latest version of breakurl and its documentation can
% be obtained at:
% http://www.ctan.org/tex-archive/macros/latex/contrib/breakurl/
% breakurl.sty is not for use under pdflatex pdf mode.
%
% The advanced features offer by hyperref.sty are not required for IEEE
% submission, so users should weigh these features against the added
% complexity of use.
% The package options above demonstrate how to enable PDF bookmarks
% (a type of table of contents viewable in Acrobat Reader) as well as
% PDF document information (title, subject, author and keywords) that is
% viewable in Acrobat reader's Document_Properties menu. PDF document
% information is also used extensively to automate the cataloging of PDF
% documents. The above set of options ensures that hyperlinks will not be
% colored in the text and thus will not be visible in the printed page, % but will be active on "mouse over". USING COLORS OR OTHER HIGHLIGHTING
% OF HYPERLINKS CAN RESULT IN DOCUMENT REJECTION BY THE IEEE, especially if
% these appear on the "printed" page. IF IN DOUBT, ASK THE RELEVANT
% SUBMISSION EDITOR. You may need to add the option hypertexnames=false if
% you used duplicate equation numbers, etc., but this should not be needed
% in normal IEEE work.
% The latest version of hyperref and its documentation can be obtained at:
% http://www.ctan.org/tex-archive/macros/latex/contrib/hyperref/





% *** Do not adjust lengths that control margins, column widths, etc. ***
% *** Do not use packages that alter fonts (such as pslatex).     ***
% There should be no need to do such things with IEEEtran.cls V1.6 and later.
% (Unless specifically asked to do so by the journal or conference you plan
% to submit to, of course. )


% correct bad hyphenation here
\hyphenation{op-tical net-works semi-conduc-tor}


\begin{document}
%
% paper title
% can use linebreaks \\ within to get better formatting as desired
% Do not put math or special symbols in the title.
\title{Memora: A Lifelog Search Engine}
%
%
% author names and IEEE memberships
% note positions of commas and nonbreaking spaces ( ~ ) LaTeX will not break
% a structure at a ~ so this keeps an author's name from being broken across
% two lines.
% use \thanks{} to gain access to the first footnote area
% a separate \thanks must be used for each paragraph as LaTeX2e's \thanks
% was not built to handle multiple paragraphs
%
%
%\IEEEcompsocitemizethanks is a special \thanks that produces the bulleted
% lists the Computer Society journals use for "first footnote" author
% affiliations. Use \IEEEcompsocthanksitem which works much like \item
% for each affiliation group. When not in compsoc mode, % \IEEEcompsocitemizethanks becomes like \thanks and
% \IEEEcompsocthanksitem becomes a line break with idention. This
% facilitates dual compilation, although admittedly the differences in the
% desired content of \author between the different types of papers makes a
% one-size-fits-all approach a daunting prospect. For instance, compsoc 
% journal papers have the author affiliations above the "Manuscript
% received ..." text while in non-compsoc journals this is reversed. Sigh.

\author{Blake~Elias, MIT}% <-this % stops a space


% note the % following the last \IEEEmembership and also \thanks - 
% these prevent an unwanted space from occurring between the last author name
% and the end of the author line. i.e., if you had this:
% 
% \author{....lastname \thanks{...} \thanks{...} }
%           ^------------^------------^----Do not want these spaces!
%
% a space would be appended to the last name and could cause every name on that
% line to be shifted left slightly. This is one of those "LaTeX things". For
% instance, "\textbf{A} \textbf{B}" will typeset as "A B" not "AB". To get
% "AB" then you have to do: "\textbf{A}\textbf{B}"
% \thanks is no different in this regard, so shield the last } of each \thanks
% that ends a line with a % and do not let a space in before the next \thanks.
% Spaces after \IEEEmembership other than the last one are OK (and needed) as
% you are supposed to have spaces between the names. For what it is worth, % this is a minor point as most people would not even notice if the said evil
% space somehow managed to creep in.



% The paper headers
\markboth{MIT 6.867 Introduction to Machine Learning Class Projects, December~2014}%
{Shell \MakeLowercase{\textit{et al.}}: Bare Advanced Demo of IEEEtran.cls for Journals}
% The only time the second header will appear is for the odd numbered pages
% after the title page when using the twoside option.
% 
% *** Note that you probably will NOT want to include the author's ***
% *** name in the headers of peer review papers.          ***
% You can use \ifCLASSOPTIONpeerreview for conditional compilation here if
% you desire.



% The publisher's ID mark at the bottom of the page is less important with
% Computer Society journal papers as those publications place the marks
% outside of the main text columns and, therefore, unlike regular IEEE
% journals, the available text space is not reduced by their presence.
% If you want to put a publisher's ID mark on the page you can do it like
% this:
%\IEEEpubid{0000--0000/00\$00.00~\copyright~2012 IEEE}
% or like this to get the Computer Society new two part style.
%\IEEEpubid{\makebox[\columnwidth]{\hfill 0000--0000/00/\$00.00~\copyright~2012 IEEE}%
%\hspace{\columnsep}\makebox[\columnwidth]{Published by the IEEE Computer Society\hfill}}
% Remember, if you use this you must call \IEEEpubidadjcol in the second
% column for its text to clear the IEEEpubid mark (Computer Society journal
% papers don't need this extra clearance.)



% use for special paper notices
%\IEEEspecialpapernotice{(Invited Paper)}



% for Computer Society papers, we must declare the abstract and index terms
% PRIOR to the title within the \IEEEtitleabstractindextext IEEEtran
% command as these need to go into the title area created by \maketitle.
% As a general rule, do not put math, special symbols or citations
% in the abstract or keywords.
\IEEEtitleabstractindextext{%
\begin{abstract}
Lifelogging is a form of data collection where a peron wears a computer to capture data of their entire life, typically through photo or video data and possibly through physiological data. In this project I present Memora, a software tool for making sense of the data that one captures from their entire life history.
\end{abstract}

% Note that keywords are not normally used for peerreview papers.
\begin{IEEEkeywords}
lifelog, scene recognition, classification, search engine
\end{IEEEkeywords}}


% make the title area
\maketitle


% To allow for easy dual compilation without having to reenter the
% abstract/keywords data, the \IEEEtitleabstractindextext text will
% not be used in maketitle, but will appear (i.e., to be "transported")
% here as \IEEEdisplaynontitleabstractindextext when compsoc mode
% is not selected <OR> if conference mode is selected - because compsoc
% conference papers position the abstract like regular (non-compsoc)
% papers do!
\IEEEdisplaynontitleabstractindextext
% \IEEEdisplaynontitleabstractindextext has no effect when using
% compsoc under a non-conference mode.


% For peer review papers, you can put extra information on the cover
% page as needed:
% \ifCLASSOPTIONpeerreview
% \begin{center} \bfseries EDICS Category: 3-BBND \end{center}
% \fi
%
% For peerreview papers, this IEEEtran command inserts a page break and
% creates the second title. It will be ignored for other modes.
\IEEEpeerreviewmaketitle



\section{Introduction}
\IEEEPARstart{F}{or} the past 3 months, I have been using a wearable device called the Narrative Clip (http://getnarrative.com) to capture my lifelogging data. The device contains a camera which is programmed to automatically take a photograph every 30 seconds, as well as a GPS location sensor which records its location along with every photograph.

This device and others like it promise to give people a ``photographic memory'' ability through which they will be able to relive and reflect on their life experiences. It seems plausible to expect this, since every moment that you experience will be captured and stored on a computer. Even if one photo every 30 seconds is not actually your \textit{entire} life experience, looking back at the stream of photos from a day of your life 5 years ago would give you a very good idea of what happened that day - much better than what details you'd be able to recall without the photos. Using a lifelogging device could enhance memory by jogging memories which still exist in your mind but would otherwise never get activated. It could help you regain access to memories which, while you may have ``forgotten about'' them, you haven't actually ``forgotten''.

However, the device makers have hardly made the experience of viewing this data feel like having a photographic memory. As the size of the total collection increases, the feeling becomes more and more like having a barely-navigable list of files on a computer that you'd have to sift through endlessly in order to find the thing you're interested in. The time it would take you to find the precise moment you care about from a year's worth of photographs (about half a million of them), depending how well you're able to remember the specific date and time to look up, would likely make the experience overwhelming. If the task is too hard for one to even want to try, the experience of having a lifelog would be unexciting.

On the other hand, within this data set lie many special memories from our lives, exciting experiences worth sharing with friends and future generations, and possible insights into how we can live better. If we hope to turn a pile of photos into a machine-enhanced super-power of photographic memory, then we need software tools for browsing these photographs in a way that is more intuitive to humans, rather than computers. The computer's easiest querying scheme is based on the name of the file, the day it was taken on, and other relatively generic, unemotional features. The human's easiest querying scheme, which we use for our own memory, is a lot more personal: where we were, who we were with, what we were doing, and so on. When you and a close friend want to talk about an experience you shared some time in the past, you bring up the memory in your friend's mind by telling a story with a bunch of features, possibly in chronological sequence. As a simple example, ``remember the time when we were with Sally in her apartment before going to the Red Sox game?'' The listener almost instantly knows what you are talking about and has some image, phrase, or other experience now re-playing in their head. They are able to perform this task because their mind has stored the memory in a way that is retrievable with the type of language you've used to describe it. Computer programs don't yet store memories in the same way that we do, but in this project I demonstrate a way that an algorithm can begin to match our intuitive structure of memory. Using Memora, I can ask the computer ``show me the first lecture of my 6.867 class'', and with some supervised training, have it understand which moments match my query.

\section{Feature Selection}

I assign features to photographs based on whose faces are present in a photo. Each point (a photo is one point) is assigned a bit-vector of people who are present in the photograph. If there is a total of $N$ people whose faces appear in my dataset, each photograph's feature vector is a length-$N$ bit vector where a person's slot contains a 1 if their face is recognized in the photo, and a 0 if not. I was able to recognize faces using Facebook's face recognition algorithm as well as the one in scikit-learn, however I did not use the results from these algorithms in my final implementation due to some implementation challenges. I hand-labeled faces in the photographs I wanted to classify, realizing that with more time spent on the engineering of this system the faces could just as well have been labeled by a machine.


\section{Activity Classification}

The first problem that I solved is training an algorithm to recognize common scenes that I experience regularly. For example, the algorithm should be able to differentiate between times I've been in one class versus another, versus when I'm at home with my family.

I have solved this using supervised learning: I select scenes from the timeline and mark them with particular labels - eg. a 1.5-hour block for a 6.867 lecture, a 1-hour block for another class, and a 15-minute block at home with my friends. Since each 30-second interval is one point in the original set, I combine features from longer blocks by taking the union of the set of faces recognized in each photo (for this classifier, I used faces as the only features in the classifier), so that a long scene has a feature set that looks like any normal point. Then, I run a nearest-neighbors classifier to classify every photograph in the timeline. I used scikit-learn's built-in $k$-Nearest-Neighbors algorithm with $k=1$ (each point gets the same label as its nearest neighbor), with Jaccard distance as the distance metric (number of dimensions where the values are non-equal in the two bit-vectors, divided by the number of dimensions that are non-zero in at least one of the two vectors).

This algorithm worked very well, but with a limited scope. $1$-NN with Jaccard distance correctly classifies all the photographs which contain at least one face that is also found in one of the labeled scenes. Photos that have only new faces (people who are not in any labeled scene) get assigned to an arbitrary class, which is wrong. Furthermore, nearest-neighbors cannot classify photographs that have no faces in them, since the bit-vector is all 0s - I had my algorithm skip these, but if they were classified they would also end up in an arbitrary, wrong class (I'll come back to this in the ``Activity Transitions'' section). But let's ignore these problems for a moment. Even the ability to classify photos that do have faces in them, while it worked well in my small example, would not necessarily be robust to all situations - for example, there was no person in my data set who showed up in two scenes that had different labels. Had there been any overlap of this sort, the algorithm would likely start making mistakes. It could see a photo with one friend of mine and classify it as either a class we have together or dinner with a group of friends who we often eat with. Solving this would likely require using a higher value of $k$ - that is, to use more neighbors in the classification.

%% TODO: show graphs of labeled timeline sections and auto-classified ones


% \subsection{Feature Smoothing}

%% TODO: talk about feature smoothing

\section{Activity Transitions}

Next, I tried using a Gaussian HMM to infer a scene based on the faces in each photo, without training on any labeled data. Here, the hidden variable is the scene label, and the observation is a bit-vector of all the faces that were recognized in each photo. I filtered for photographs that had faces in them, and then passed these as observations to the HMM. Since the HMM is unsupervised, it does not know any examples of which photos belong to which category - the only information it was given was the number of different values that the hidden state should be allowed to take on. I took a subset of the images I had labeled, counted the number of different labels I had given to the photos, and then added 1 to allow for the possibility of a state that represents all the `in-between' photos - photos that don't belong to any of the scene classes but instead lie in transitions from one activity to the next - things like walking in a hallway, riding a train and so on. In the subset of images I chose, there had been three labeled activities so I set the number of states equal to 4. I estimated an initial value of the transition matrix as follows:

\[
\left( \begin{array}{cccc}
0.7 & 0.1 & 0.1 & 0.1 \\
0.1 & 0.7 & 0.1 & 0.1 \\
0.1 & 0.1 & 0.7 & 0.1 \\
0.1 & 0.1 & 0.1 & 0.7
\end{array}
\right) 
\]

These initial values were appropriate in order to make the hidden state usually stay where it is (because I don't typically switch back and forth between scenes many times - from minute to minute, I usually stay in the same activity like a class, and then once every hour or two I will move to a new class or activity).

I fed the observations and my initial guess of the transition matrix into the Gaussian HMM provided by scikit-learn, and used its built-in ``fit'' procedure to estimate the model parameters and hidden states using EM. After running it $1000$ times, I selected the model with the highest score and found a sequence of hidden states which only misclassified 4 photos out of 40 from my original scene labeling (that is to say, photos which I had marked as belonging to one class (though the HMM didn't know that), and which the HMM put in another class). Though it isn't defined which hidden state in the HMM corresponds to which label in my original labeling, it is pretty easy to tell by inspection when looking at the results, as photos with the same original label mostly received the same hidden state from the HMM, and each different label got a different hidden state as the most common state among its photos.

What did not work so well was the states assigned to the images I had not labeled. These photos have miscellaneous faces in them who were not relevant to any of the activities that I had labeled. Only 3 of the non-labeled images were assigned to the ``other'' state, out of 23 - the rest were assigned a hidden state that was the same as the state for one of the labeled scene categories.

\section{Results Illustration}

Manually labeled classes:

\begin{verbatim}
[A A A A A A A A A A - - - - 
 B B B B B B B B B B B - - -
 - - - - - - - - - - - - - -
 - - C C C C C C C C C C C C
 C C C C C C C]
\end{verbatim}

Hidden state assignments in the HMM, unsupervised:

\begin{verbatim}
[1 1 1 1 1 1 1 1 1 1 1 1 1 1
 3 3 3 3 3 3 3 3 3 3 3 3 0 0
 0 1 1 1 1 1 3 3 3 3 3 3 3 3
 3 3 1 2 2 2 2 2 2 1 1 2 2 2
 2 2 2 2 2 1 2]
\end{verbatim}

% An example of a floating figure using the graphicx package.
% Note that \label must occur AFTER (or within) \caption.
% For figures, \caption should occur after the \includegraphics.
% Note that IEEEtran v1.7 and later has special internal code that
% is designed to preserve the operation of \label within \caption
% even when the captionsoff option is in effect. However, because
% of issues like this, it may be the safest practice to put all your
% \label just after \caption rather than within \caption{}.
%
% Reminder: the "draftcls" or "draftclsnofoot", not "draft", class
% option should be used if it is desired that the figures are to be
% displayed while in draft mode.
%
%\begin{figure}[!t]
%\centering
%\includegraphics[width=2.5in]{myfigure}
% where an .eps filename suffix will be assumed under latex, 
% and a .pdf suffix will be assumed for pdflatex; or what has been declared
% via \DeclareGraphicsExtensions.
%\caption{Simulation Results.}
%\label{fig_sim}
%\end{figure}

% Note that IEEE typically puts floats only at the top, even when this
% results in a large percentage of a column being occupied by floats.
% However, the Computer Society has been known to put floats at the bottom.


% An example of a double column floating figure using two subfigures.
% (The subfig.sty package must be loaded for this to work.)
% The subfigure \label commands are set within each subfloat command, % and the \label for the overall figure must come after \caption.
% \hfil is used as a separator to get equal spacing.
% Watch out that the combined width of all the subfigures on a 
% line do not exceed the text width or a line break will occur.
%
%\begin{figure*}[!t]
%\centering
%\subfloat[Case I]{\includegraphics[width=2.5in]{box}%
%\label{fig_first_case}}
%\hfil
%\subfloat[Case II]{\includegraphics[width=2.5in]{box}%
%\label{fig_second_case}}
%\caption{Simulation results.}
%\label{fig_sim}
%\end{figure*}
%
% Note that often IEEE papers with subfigures do not employ subfigure
% captions (using the optional argument to \subfloat[]), but instead will
% reference/describe all of them (a), (b), etc., within the main caption.


% An example of a floating table. Note that, for IEEE style tables, the 
% \caption command should come BEFORE the table. Table text will default to
% \footnotesize as IEEE normally uses this smaller font for tables.
% The \label must come after \caption as always.
%
%\begin{table}[!t]
%% increase table row spacing, adjust to taste
%\renewcommand{\arraystretch}{1.3}
% if using array.sty, it might be a good idea to tweak the value of
% \extrarowheight as needed to properly center the text within the cells
%\caption{An Example of a Table}
%\label{table_example}
%\centering
%% Some packages, such as MDW tools, offer better commands for making tables
%% than the plain LaTeX2e tabular which is used here.
%\begin{tabular}{|c||c|}
%\hline
%One & Two\\
%\hline
%Three & Four\\
%\hline
%\end{tabular}
%\end{table}


% Note that IEEE does not put floats in the very first column - or typically
% anywhere on the first page for that matter. Also, in-text middle ("here")
% positioning is not used. Most IEEE journals use top floats exclusively.
% However, Computer Society journals sometimes do use bottom floats - bear
% this in mind when choosing appropriate optional arguments for the
% figure/table environments.
% Note that, LaTeX2e, unlike IEEE journals, places footnotes above bottom
% floats. This can be corrected via the \fnbelowfloat command of the
% stfloats package.


\section{Contributions}

 (1) A scene-classification model that classifies scenes using labeled examples, using face recognition to generate the features. 

 (2) Another scene-recognition approach which can group images together into scenes with no labeled training examples at all. Unlike other common scene-recognition approaches, this approach takes advantage of the similarity between images that were taken at around the same time, rather than just using the features of one image.

\section{Discussion}

While these models don't fully solve the problem of recognizing activities from daily life, they show some promising results that such a task is feasible to be done by computers with relatively straightforward application of machine learning techniques. 

The nearest-neighbors classifier could be useful when one is browsing lifelog photos and wants to see all scenes of a particular category, where they have told the machine about some other scenes of that category before. The HMM approach could be used for gathering photos into scenes without the machine necessarily knowing what the scenes are. This would be useful in a photo-browsing interface to facilitate tagging of scenes, since most of the images belonging to a scene could already be grouped together, rather than the user having the tedious task of locating boundaries between scenes. The HMM could also be useful in a photo-browsing interface to reduce redundancy of similar photos. For example, if one wants to play through an entire day's worth of photos (1000-2000 photos), a browsing interface could animate the photos as frames at a high speed, increasing the speed when a range of consecutive images all have the same hidden state, and slowing down the speed during transitions from one state to another. 

\section{Future Work}

I plan to try the following possibilities to improve the accuracy of both of these models. First is adding more features from the data that I have. GPS coordinates are available from the Narrative Clip at every moment when a photo was taken. This would help a great deal in differentiating photos from different scenes, since moving from one location to another one is highly likely to represent a transition from one activity to another (extraneous movement might have to be removed for things like going to the bathroom in the middle of a class, if the end-user does not want this classified as a separate activity), or as an activity in itself (such as running, biking, or moving due to transportation). (The only reason I did not use GPS in this paper was that the Clip doesn't fully compute the coordinates, it just records some measurements to save battery on the device, and leaves it up to the developer to map these measurements to coordinates - I opted not to do investigate this in the interest of time, but it certainly would help.) I would also consider adding other image features, such as values from the color histogram - I believe that this would readily detect changes in environment and help group images together that belong to the same scene. This would reduce some of the extraneous transitions that the HMM generated when faces would come and go, since the histogram would remain largely unchanged due to the constant background in the images. Other features worth adding would be object labels using an object-recognition library - this would allow for classification based on objects you've seen, like furniture, plates and silverware, and so on.

Another future technique to try is to smooth out the appearance of features using a moving window. Specifically, if a person's face shows up in a photo, then is absent in the next five photos but is present again in the next photo, my current model encodes that sequence as [... 1 0 0 0 0 0 1 ...], which makes no indication that the person was present in the five middle photos. Instead, this could be encoded as something like [... 1 0.7 0.5 0.4 0.5 0.7 1], which would provide some indication that the person was present in those in-between moments. This would help remove some of the extraneous transitions of the HMM, and could even help with the kNN classifier since it would be able to synthesize the concept of a group of people being together during some time range even if no one photo contains all of those people at the same time.

% if have a single appendix:
%\appendix[Proof of the Zonklar Equations]
% or
%\appendix % for no appendix heading
% do not use \section anymore after \appendix, only \section*
% is possibly needed

% use appendices with more than one appendix
% then use \section to start each appendix
% you must declare a \section before using any
% \subsection or using \label (\appendices by itself
% starts a section numbered zero.)
%





% trigger a \newpage just before the given reference
% number - used to balance the columns on the last page
% adjust value as needed - may need to be readjusted if
% the document is modified later
%\IEEEtriggeratref{8}
% The "triggered" command can be changed if desired:
%\IEEEtriggercmd{\enlargethispage{-5in}}

% references section

% can use a bibliography generated by BibTeX as a .bbl file
% BibTeX documentation can be easily obtained at:
% http://www.ctan.org/tex-archive/biblio/bibtex/contrib/doc/
% The IEEEtran BibTeX style support page is at:
% http://www.michaelshell.org/tex/ieeetran/bibtex/
%\bibliographystyle{IEEEtran}
% argument is your BibTeX string definitions and bibliography database(s)
%\bibliography{IEEEabrv,../bib/paper}
%
% <OR> manually copy in the resultant .bbl file
% set second argument of \begin to the number of references
% (used to reserve space for the reference number labels box)


% biography section
% 
% If you have an EPS/PDF photo (graphicx package needed) extra braces are
% needed around the contents of the optional argument to biography to prevent
% the LaTeX parser from getting confused when it sees the complicated
% \includegraphics command within an optional argument. (You could create
% your own custom macro containing the \includegraphics command to make things
% simpler here.)
%\begin{IEEEbiography}[{\includegraphics[width=1in,height=1.25in,clip,keepaspectratio]{mshell}}]{Michael Shell}
% or if you just want to reserve a space for a photo:


% You can push biographies down or up by placing
% a \vfill before or after them. The appropriate
% use of \vfill depends on what kind of text is
% on the last page and whether or not the columns
% are being equalized.

%\vfill

% Can be used to pull up biographies so that the bottom of the last one
% is flush with the other column.
%\enlargethispage{-5in}



% that's all folks
\end{document}


